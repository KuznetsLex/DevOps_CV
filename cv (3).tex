%-----------------------------------------------------------------------------------------------------------------------------------------------%
%	The MIT License (MIT)
%
%	Copyright (c) 2021 Jitin Nair
%
%	Permission is hereby granted, free of charge, to any person obtaining a copy
%	of this software and associated documentation files (the "Software"), to deal
%	in the Software without restriction, including without limitation the rights
%	to use, copy, modify, merge, publish, distribute, sublicense, and/or sell
%	copies of the Software, and to permit persons to whom the Software is
%	furnished to do so, subject to the following conditions:
%	
%	THE SOFTWARE IS PROVIDED "AS IS", WITHOUT WARRANTY OF ANY KIND, EXPRESS OR
%	IMPLIED, INCLUDING BUT NOT LIMITED TO THE WARRANTIES OF MERCHANTABILITY,
%	FITNESS FOR A PARTICULAR PURPOSE AND NONINFRINGEMENT. IN NO EVENT SHALL THE
%	AUTHORS OR COPYRIGHT HOLDERS BE LIABLE FOR ANY CLAIM, DAMAGES OR OTHER
%	LIABILITY, WHETHER IN AN ACTION OF CONTRACT, TORT OR OTHERWISE, ARISING FROM,
%	OUT OF OR IN CONNECTION WITH THE SOFTWARE OR THE USE OR OTHER DEALINGS IN
%	THE SOFTWARE.
%	
%
%-----------------------------------------------------------------------------------------------------------------------------------------------%

%----------------------------------------------------------------------------------------
%	DOCUMENT DEFINITION
%----------------------------------------------------------------------------------------

% article class because we want to fully customize the page and not use a cv template
\documentclass[a4paper,12pt]{article}

%----------------------------------------------------------------------------------------
%	FONT
%----------------------------------------------------------------------------------------

% % fontspec allows you to use TTF/OTF fonts directly
% \usepackage{fontspec}
% \defaultfontfeatures{Ligatures=TeX}

% % modified for ShareLaTeX use
% \setmainfont[
% SmallCapsFont = Fontin-SmallCaps.otf,
% BoldFont = Fontin-Bold.otf,
% ItalicFont = Fontin-Italic.otf
% ]
% {Fontin.otf}

%----------------------------------------------------------------------------------------
%	PACKAGES
%----------------------------------------------------------------------------------------
\usepackage{url}
\usepackage{parskip} 	
\usepackage[utf8]{inputenc}
\usepackage[russian]{babel}

%other packages for formatting
\RequirePackage{color}
\RequirePackage{graphicx}
\usepackage[usenames,dvipsnames]{xcolor}
\usepackage[scale=0.9]{geometry}

%tabularx environment
\usepackage{tabularx}

%for lists within experience section
\usepackage{enumitem}

% centered version of 'X' col. type
\newcolumntype{C}{>{\centering\arraybackslash}X} 

%to prevent spillover of tabular into next pages
\usepackage{supertabular}
\usepackage{tabularx}
\newlength{\fullcollw}
\setlength{\fullcollw}{0.47\textwidth}

%custom \section
\usepackage{titlesec}				
\usepackage{multicol}
\usepackage{multirow}

%CV Sections inspired by: 
%http://stefano.italians.nl/archives/26
\titleformat{\section}{\Large\scshape\raggedright}{}{0em}{}[\titlerule]
\titlespacing{\section}{0pt}{10pt}{10pt}

%for publications
\usepackage[style=authoryear,sorting=ynt, maxbibnames=2]{biblatex}

%Setup hyperref package, and colours for links
\usepackage[unicode, draft=false]{hyperref}
\definecolor{linkcolour}{rgb}{0,0.2,0.6}
\hypersetup{colorlinks,breaklinks,urlcolor=linkcolour,linkcolor=linkcolour}
\addbibresource{citations.bib}
\setlength\bibitemsep{1em}

%for social icons
\usepackage{fontawesome5}

%debug page outer frames
%\usepackage{showframe}

%----------------------------------------------------------------------------------------
%	BEGIN DOCUMENT
%----------------------------------------------------------------------------------------
\begin{document}

% non-numbered pages
\pagestyle{empty} 

%----------------------------------------------------------------------------------------
%	TITLE
%----------------------------------------------------------------------------------------

% \begin{tabularx}{\linewidth}{ @{}X X@{} }
% \huge{Алексей Кузнецов}\vspace{2pt} & \hfill \emoji{incoming-envelope} email@email.com \\
% \raisebox{-0.05\height}\faGithub\ username \ | \
% \raisebox{-0.00\height}\faLinkedin\ username \ | \ \raisebox{-0.05\height}\faGlobe \ mysite.com  & \hfill \emoji{calling} number
% \end{tabularx}

\begin{tabularx}{\linewidth}{@{} C @{}}
\Huge{Алексей Кузнецов}\\[7.5pt]
\Large{DevOps}\\
\href{https://github.com/KuznetsLex}{\raisebox{-0.05\height}\faGithub\ KuznetsLex} \ $|$ \ 
\href{mailto:kuznets.work@yandex.ru}{\raisebox{-0.05\height}\faEnvelope \ kuznets.work@yandex.ru} \ $|$ \ 
\href{tel:+79835289171}{\raisebox{-0.05\height}\faMobile \ +79835289171}\\
\end{tabularx}

%----------------------------------------------------------------------------------------
% EXPERIENCE SECTIONS
%----------------------------------------------------------------------------------------

%Interests/ Keywords/ Summary
\section{Обо мне}
- Учусь на 3 курсе матфака ОмГУ, средний балл зачтеки 4,7. \\
- Прошел курсы по Linux и Docker на Stepik. Читаю Windows Networking Essentials. \\
- Опыт программирования на Python – 4 года. Опыт некоммерческой разработки Android- и iOS-приложений – 2 года. \\
- Увлекаюсь спортивным бриджем и плаванием. 

%Projects
\section{Проекты}

\begin{tabularx}{\linewidth}{ @{}l r@{} }
\textbf{Chess Puzzle Generator} & \hfill \href{https://github.com/KuznetsLex/Chess-Puzzle-Generator}{Ссылка на Демо} \\[3.75pt]
\multicolumn{2}{@{}X@{}}{2024г. Сервис для генерации схожих шахматных задач. Программа извлекает случайную задачу выбранного уровня сложности из базы данных и генерирует на её основе массив похожих задач. Сгенерированная задача отличается от исходной позициями и достоинствами фигур игрока. Я работал над алгоритмом генерации и визуализацией задач. Проект разработан на Python 3.10, использовалась библиотеки pandas для работы с базой задач.}  \\\\
{\textbf{Knapsack Problem Solver}} & \hfill \href{https://github.com/Optimization-Methods-Project-OmSU-2024/Knapsack-problem}{Ссылка на Демо} \\[3.75pt]
\multicolumn{2}{@{}X@{}}{
2024г. Консольное приложение позволяет находить оптимальное решение задачи о рюкзаке в следующих постановках: задача о булевом рюкзаке, задача о целочисленном рюкзаке, нелинейная задача о рюкзаке. Я реализовал нелинейную задачу. Программа написана на языке Java.}  \\\\
{\textbf{Shortest Path Finder}} & \hfill \href{https://github.com/Optimization-Methods-Project-OmSU-2024/Shortest-path-problem}{Ссылка на Демо} \\[3.75pt]
\multicolumn{2}{@{}X@{}}{
2024г. Консольное приложение позволяет строить кратчайшие пути в графе, используя алгоритмы Дейкстры, Форда-Беллмана и Флойда-Уоршелла. Я реализовал алгоритм Форда-Беллмана. Программа написана на языке Java.}  \\\\
\textbf{Newston} & \hfill \href{https://github.com/KuznetsLex/Newston}{Ссылка на Демо} \\[3.75pt]
\multicolumn{2}{@{}X@{}}{2023г. iOS-приложение для email-newsletters, то есть статей и газет по интересам. Я сверстал экраны и настроил подгрузку статей из базы. Проект написан на языке Swift. Выдержана структура MVVM. Для работы с http-запросами использовалась библиотека Alamofire.}  \\\\
{\textbf{AlphaBeta}} & \hfill \href{https://github.com/KuznetsLex/AlphaBeta}{Ссылка на Демо} \\[3.75pt]
\multicolumn{2}{@{}X@{}}{
2022г. Android-приложение для обучения ребенка чтению и письму букв английского алфавита. Я работал над проверкой правильности рисования буквы. Приложение написано на языке Kotlin, для разработки интерфейса использовался язык разметки XML. Для распознавания произношения и написания использовались готовые нейронные сети и технология Google ML Kit.} \\\\
{\textbf{Zootutor}} & \hfill \\[3.75pt]
\multicolumn{2}{@{}X@{}}{
2021г. Android-приложение для знакомства ребенка с фермерскими животными. Оно показывает, как животные выглядят, что едят, как "говорят". Я верстал экраны с животными, подключал аудиофайлы на страницы. Проект разработан на языке Kotlin, для разработки интерфейса использовался язык разметки XML.} \\\\
\end{tabularx}

%----------------------------------------------------------------------------------------
%	EDUCATION
%----------------------------------------------------------------------------------------
\section{Образование}
\begin{tabularx}{\linewidth}{@{}l X@{}}	
2022 - наст. время & Матфак ОмГУ им. Ф.М. Достоевского \hfill \normalsize (Средний балл зачетки: 4,7) \\
2024 & Курсы на Stepik по Linux и Docker\hfill (100\%)\\
2022 & Студенческая IT-Лаборатория, iOS-разработчик\hfill (Newston)\\
2017 - 2022 & АНО ДПО "Махаон"\hfill(Zootutor, AlphaBeta и др.) \\ 
2011-2022 & Гимназия №115 \hfill  (Красный аттестат) \\
\end{tabularx}

%Achievements
\section{Достижения}
\begin{tabularx}{\linewidth}{@{}l X@{}}	
- Диплом I степени МНСК-2022 (проект AlphaBeta) \hfill \normalsize  \\
- Диплом за 1 место в хакатоне "IT-Лидер" 2021 (проект Zootutor)\\
- Победа в XXII Кубке России по бриджу. Секция "Будущее" \hfill  \\
\end{tabularx}
%----------------------------------------------------------------------------------------
%	SKILLS
%----------------------------------------------------------------------------------------
\section{DevOps Hardskills}

\textbf{Linux}\\
Bash, SSH, Systemd, настройка firewall.\\
Траблшутинг: контроль места на диске, состояния дисков, нагрузки, процессов, сети, логов.\\

\textbf{Сети} \\
Модель OSI и TCP/IP: назначение слоев, оборудование, протоколы.\\

\textbf{Docker} \\
Архитектура, images, network, bind mounts / volumes. \\
Оптимизация Dockerfiles для сервисов на Spring, Go, Node, Nginx, Python.\\
Docker compose. \\
Docker Swarm, k8s.\\ 

\textbf{Мониторинг и алертинг}\\
Prometheus + Grafana. cAdvisor, Node Exporter.\\

\textbf{Работа в облаке}\\
Создание и подключение к виртуальной машине Yandex Cloud.\\

\section{Прочие навыки}

\begin{tabularx}{\linewidth}{@{}l X@{}}
Python, Java, Swift, Kotlin &  \normalsize{Языки, используемые в проектах.}\\\\

SQL, C, C++, Maxima &  \normalsize{Языки, используемые в лабораторных в университете.}\\\\

Английский язык (уровень C1) &  \normalsize{Окончил лингвистическую гимназию, в университете учусь в сильнейшей группе по английскому.}\\\\

Лидерские и управляющие качества  &  \normalsize {Прошел стажировку на PM в Effective. Совмещал роль разработчика с ролью PM на проектах Path Finder и Knapsack Problem.}\\ \\

Организаторские навыки  &  \normalsize {Член команды организаторов Students Lab-2023 и DevFest-2024.}\\ \\

\end{tabularx}

\vfill
\center{\footnotesize Обновлено: \today}

\end{document}
